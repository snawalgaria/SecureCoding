\chapter{Tools}\label{chapter:tools}

\section{Zed Attack Proxy (ZAP)}
Using the Zed Attack Proxy (ZAP), we were able to reveal significant parts of the directory structure in both web applications. In the \textit{Online Banking} web application, we found a stored XSS vulnerability in the registration and the transaction description as well as a SQL injection vulnerability in the login form using the fuzzer. We were also be able to find a buffer overflow vulnerability for the transaction description in the transaction batch files. In the \textit{SecureBank} web application, we were unable to find further SQL injection or XSS vulnerabilities.

\section{SQLmap}
Using SQLmap, we found the SQL injection vulnerability in the login form of the \textit{Online Banking} application, which we found using ZAP earlier. SQLmap did not reveal further SQL injection possibilities in any of the two applications.

\section{W3AF}
With W3AF, we made a full audit, including file\_upload(looks for uploadable files), eval(insecure eval() usage), un\_ssl(secure content via http), os\_commanding, lfi(local file inclusion), rfi, sqli(injection), preg\_replace(insecure preg\_replace() in PHP), mx\_injection, generic, format\_string, websocket\_hijacking, shell\_shock, ldapi, buffer\_overflow, redos(DOS using slow regex), global\_redirect(any redirecting scripts), xpath, cors\_origin(consistency of HTTP origin header and sender), htaccess\_methods, dav(WebDAV module configuration), ssi(server side inclusion), csrf, xss, ssl\_certificate, xst(cross site tracing), blind\_sqli, phishing\_vector, response\_splitting, rfd(reflected file download), frontpage(tries uploading files using frontpage extensions).

W3AF was able to quite easily find XSS, CSRF and SQL injection points. It also informed about unhandled errors and possible click-jacking on both sites.

However, it also showed a few false positives (showing sql injections as eval vulnerabilities, path disclosure) and missed a few vulnerabilities that our other tools found, like buffer overflows.

Overall, I'd say W3AF is a Jack of all Trades. If the functionality is not enough, however, it's possible to write custom plugins.

\section{SQL Inject Me}

Already installed in Firefox on the Samurai machines, this tool has one goal and is pretty straightforward to use. It attacks the chosen form fields with brute-force SQL Injection attacks, hoping to get some intel on possible attack vectors. Fields that you don't want to attack can be supplied with a fixed value.
Attacking the username in the login form of Online Banking passes 15479 of 15480 tests, but the input name \texttt{' OR username IS NOT NULL OR username = '} returns a 302 error code. This shows us that the username field might be vulnerable.

However, seeing that only one test shows some kind of reaction in a place where we afterwards confirmed the existence of a rather simple vulnerability (\texttt{admin'\#}) makes "SQL Inject Me" seem like a rather weak tool.

Something else to behold is that when brute-forcing any kind of form where data is created (like registration), there will be more than 15.000 similar actions performed within a few seconds. The tool doesn't do anything to hide its intentions.

