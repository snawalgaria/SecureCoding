\chapter{Tools}\label{chapter:tools}

\section{Zed Attack Proxy (ZAP)}
Using the Zed Attack Proxy (ZAP), we were able to reveal significant parts of the directory structure in both web applications. In the \textit{Online Banking} web application, we found a stored XSS vulnerability in the registration and the transaction description as well as a SQL injection vulnerability in the login form using the fuzzer. We were also be able to find a buffer overflow vulnerability for the transaction description in the transaction batch files. In the \textit{SecureBank} web application, we were unable to find further SQL injection or XSS vulnerabilities.

\section{SQLmap}
<<<<<<< HEAD
Using SQLmap, we found the SQL injection vulnerability in the login form, which we found using ZAP earlier. SQLmap did not reveal further SQL injection possibilities.

\section{BURP}
Using Burp, we could perform SQL injection , while trying to login to the InternetBanking webApplication by giving the username as admin';#. We could also many important information of the client and the server using Burp. Information on HTTP Methods, Server, Framework etc

\section{FireBug}
FireBug was used for client side testing. We got client side scripts, html , css source code using firebug
=======
Using SQLmap, we found the SQL injection vulnerability in the login form of the \textit{Online Banking} application, which we found using ZAP earlier. SQLmap did not reveal further SQL injection possibilities in any of the two applications.

\section{W3AF}
With W3AF, we made a full audit, including file\_upload(looks for uploadable files), eval(insecure eval() usage), un\_ssl(secure content via http), os\_commanding, lfi(local file inclusion), rfi, sqli(injection), preg\_replace(insecure preg\_replace() in PHP), mx\_injection, generic, format\_string, websocket\_hijacking, shell\_shock, ldapi, buffer\_overflow, redos(DOS using slow regex), global\_redirect(any redirecting scripts), xpath, cors\_origin(consistency of HTTP origin header and sender), htaccess\_methods, dav(WebDAV module configuration), ssi(server side inclusion), csrf, xss, ssl\_certificate, xst(cross site tracing), blind\_sqli, phishing\_vector, response\_splitting, rfd(reflected file download), frontpage(tries uploading files using frontpage extensions).

W3AF was able to quite easily find XSS, CSRF and SQL injection points. It also informed about unhandled errors and possible click-jacking on both sites.

However, it also showed a few false positives (showing sql injections as eval vulnerabilities, path disclosure) and missed a few vulnerabilities that our other tools found, like buffer overflows.

Overall, I'd say W3AF is a Jack of all Trades. If the functionality is not enough, however, it's possible to write custom plugins.
>>>>>>> 2865d141a146c7c556afeef725d59bb26d5a6be0
