\chapter{Detailed Report}\label{chapter:details}

\section{Configuration and Deploy Management Testing}

\subsection{Test File Extensions Handling for Sensitive Information}

\subsubsection*{Online Banking}

\begin{tabular}{l|p{10cm}}

\textbf{Observation} & We found various files which are served as plain text but are PHP source files. One of these files contains the credentials of the mail server. We were also able to download the compiled executable as well as the source code of the batch file parser. \\
\textbf{Discovery} & Using the OWASP ZAP tool, we used the forced browse functionality on \texttt{/InternetBanking/}. We received a list of files which were found using this tool, see below. \\
\textbf{Likelihood} & This can be tested by anyone who enters specific strings into the address bar of a browser. However, the likelihood of this vulnerability is much higher if the attacker uses specific tools which test specific paths systematically. \\
\textbf{Impact} & The attacker can get sensitive information, e.g. credentials to the mail server or the database. He can analyze the source of the parser and find vulnerabilies there. \\
\textbf{Access Vector} & Network \\
\textbf{Access Complexity} & Low \\
\textbf{Privileges Required} & None \\
\textbf{User Interaction} & None \\
\textbf{Scope} & Unchanged \\
\textbf{Confidentiality} & High \\
\textbf{Intigrity} & No Impact \\
\textbf{Availability} & No Impact \\
\end{tabular}

TODO: Forced browsing results.

\subsubsection*{SecureBank}

\begin{tabular}{l|p{10cm}}

\textbf{Observation} & We found some HTML snippets, which do not contain any sensitive information, and the compiled executable of the transaction file parser. \\
\textbf{Discovery} & Using the OWASP ZAP tool, we used the forced browse functionality on \texttt{/seccoding-2015/}. We received a list of files which were found using this tool, see below. \\
\textbf{Likelihood} & This can be tested by anyone who enters specific strings into the address bar of a browser. However, the likelihood of this vulnerability is much higher if the attacker uses specific tools which test specific paths systematically. \\
\textbf{Impact} & The attacker only has access to the parser executable, which might contain information about the database connection. He can analyze the parser and find vulnerabilies there. \\
\textbf{Access Vector} & Network \\
\textbf{Access Complexity} & Low \\
\textbf{Privileges Required} & None \\
\textbf{User Interaction} & None \\
\textbf{Scope} & Unchanged \\
\textbf{Confidentiality} & Low \\
\textbf{Intigrity} & No Impact \\
\textbf{Availability} & No Impact \\
\end{tabular}

TODO: Forced browsing results.

\subsubsection*{Comparison}
The web application of the SecureBank discloses less sensitive information. However, both applications disclose information which should not be available to unauthorized persons.

\clearpage

\subsection{Test HTTP Methods}

\subsubsection*{Online Banking}

\begin{tabular}{l|p{10cm}}

\textbf{Observation} & The server responded that the method \texttt{POST}, \texttt{GET}, \texttt{OPTIONS} and \texttt{HEAD} are supported.  \\
\textbf{Discovery} & We submitted the request \texttt{OPTIONS / HTTP/1.1} to the server via NetCat on port 80. \\
\textbf{Impact} & n/a \\
\textbf{Likelihood} & n/a \\
\textbf{CVSS} & n/a \\
\end{tabular}


\subsubsection*{SecureBank}

\begin{tabular}{l|p{10cm}}

\textbf{Observation} & The server responded that the method \texttt{POST}, \texttt{GET}, \texttt{OPTIONS} and \texttt{HEAD} are supported.  \\
\textbf{Discovery} & We submitted the request \texttt{OPTIONS / HTTP/1.1} to the server via NetCat on port 80. \\
\textbf{Impact} & n/a \\
\textbf{Likelihood} & n/a \\
\textbf{CVSS} & n/a \\
\end{tabular}

\subsubsection*{Comparison}
There are no significant differences between both applications.


\subsection{Test HTTP Strict Transport Security}

\subsubsection*{Online Banking}

\begin{tabular}{l|p{10cm}}

\textbf{Observation} & The server did not send any \texttt{Strict-Transport-Security} header.  \\
\textbf{Discovery} & Executing the command \texttt{curl -s -D- http://vm/InternetBanking/ | grep Strict} resulted in no results. \\
\textbf{Impact} & n/a \\
\textbf{Likelihood} & n/a \\
\textbf{CVSS} & n/a \\
\end{tabular}


\subsubsection*{SecureBank}

\begin{tabular}{l|p{10cm}}

\textbf{Observation} & The server did not send any \texttt{Strict-Transport-Security} header.  \\
\textbf{Discovery} & Executing the command \texttt{curl -s -D- http://vm/InternetBanking/ | grep Strict} resulted in no results. \\
\textbf{Impact} & n/a \\
\textbf{Likelihood} & n/a \\
\textbf{CVSS} & n/a \\
\end{tabular}

\subsubsection*{Comparison}
There are no significant differences between both applications.

\subsection{Test RIA cross domain policy}

\subsubsection*{Online Banking}

\begin{tabular}{l|p{10cm}}

\textbf{Observation} & No cross domain policy files were found.  \\
\textbf{Discovery} & We scanned the traffic using ZAP. \\
\textbf{Impact} & n/a \\
\textbf{Likelihood} & n/a \\
\textbf{CVSS} & n/a \\
\end{tabular}

\subsubsection*{SecureBank}

\begin{tabular}{l|p{10cm}}

\textbf{Observation} & No cross domain policy files were found.  \\
\textbf{Discovery} & We scanned the traffic using ZAP. \\
\textbf{Impact} & n/a \\
\textbf{Likelihood} & n/a \\
\textbf{CVSS} & n/a \\
\end{tabular}

\subsubsection*{Comparison}
There are no significant differences between both applications.
